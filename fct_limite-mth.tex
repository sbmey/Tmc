\begin{mth}
On considère la fonction $f$ définie sur $\R$ par $f(x)=-2x^3+3x^2-10x+1$.

$\Crbf{f}$ a-t-elle une asymptote horizontale au voisinage de $-\infty$ ? et en  $+\infty$ ?
\end{mth}
%
%
%
\begin{mth}
On considère la fonction $f$ définie sur $\R$ par $f(x)=\frac{x}{x^2+1}+2$.
    \begin{enumerate}
    \item \'Etudier les limites de $f$ lorsque $x$ tend vers $-\infty$ puis en $+\infty$.
    \item Interpréter graphiquement ces résultats.
    \end{enumerate}
\end{mth}
%
%
%
\begin{mth}
On considère la fonction $f$ définie sur $\R$ par $f(x)=\frac{x+2}{x^2+1}$.
    \begin{enumerate}
    \item \'Etudier les limites de $f$ aux bornes de son ensemble de définition.
    \item Interpréter graphiquement ces résultats.
    \end{enumerate}
\end{mth}
%
%
%
\begin{mth}
On considère la fonction $f$ définie sur $\R\setminus\{-1\}$ par $f(x)=\frac{3-x^2}{x+1}$.
    \begin{enumerate}
    \item 
        \begin{enumerate}
        \item \'Etudier les limites de $f$ en $\infty$ et en $+\infty$. 
        \item Interpréter graphiquement ces résultats.
        \end{enumerate}
    \item 
            \begin{enumerate}
        \item \'Etudier la limite de $f$ lorsque $x$ tend vers $-1$. 
        \item La courbe de $f$ admet-elle une asymptote verticale en $-1$ ?
        \end{enumerate}
    \end{enumerate}
\end{mth}
%
%
%
\begin{mth}Soit $f$ la fonction définie sur $\R$ par $f(x)=4x-5\Exp^{x}$.
	\begin{enumerate}
	\item \'Etudier la limite de $f$ en $-\infty$.
	\item \'Etudier la limite de $f$ en $+\infty$.
	\end{enumerate}
\end{mth}
%
%
%
\begin{mth}Soit $f$ la fonction définie sur $\R$ par $f(x)=\mleft(-x^2+5x+10\mright)\Exp^{x}$.
	\begin{enumerate}
	\item \'Etudier la limite de $f$ en $-\infty$.
	\item \'Etudier la limite de $f$ en $+\infty$.
	\end{enumerate}
\end{mth}
%
%
%
\begin{mth}On considère la fonction $f$ définie sur $\R+*$ par $f(x)=2x+x\sin \mleft(x^2+1\mright)$.
	\begin{enumerate}
	\item Prouver : 
		\begin{enumerate}
		\item $f(x)\geqslant x$ pour tout réel $x\geqslant 0$.
		\item $f(x)\leqslant x$ pour tout réel $x\leqslant 0$.
		\end{enumerate}
	\item En déduire $\limite{x\to-\infty}{f(x)}$ et  $\limite{x\to+\infty}{f(x)}$.
	\end{enumerate}
\end{mth}
%
%
%
\begin{mth}Soit $f$ la fonction définie sur $\R*$ par $f(x)=\frac{\cos x}{x}$.

\'Etudier la limite de $f$ en $0$.
\end{mth}
%
%
%
\endinput

\begin{mth} Soit $f$ la fonction définie sur $\Retoile$ par $f(x)=\frac{x^2+x}{x^2}$.

Prouver que la droite $(d)$ d'équation $y=1$ est asymptote horizontale à $\crbf$ au voisinage de $+\infty$.
\end{mth}

\begin{mth}
\begin{enumerate}
\item Prouver que $\lim_{x\to +\infty}\frac 1x=0$ en utilisant la définition de la limite correspondante.
\item Interpréter graphiquement ce résultat.
\end{enumerate}
%\begin{enumerate}
%\item Soit $\epsilon >0$.
%
%{\itshape Aide : il faut montrer qu'il existe un réel $A$ tel que pour tout $x>A$, $-\epsilon<\frac1x<\epsilon$.
%
%$x$ tend vers $+\infty$ donc on peut se limiter aux valeurs positives de $x$. 
%
%On a alors $\frac1x<\epsilon$ lorsque $x<0$ (ce qui est inutile) ou $x>\frac1\epsilon$.
%
%On peut donc rédiger la solution : 
%}
%
%Pour tout réel $ x>\frac 1 \epsilon$, on a  $ 0< \frac 1x< \epsilon$ d'où $ \frac 1x \in \intervalleo{ -\epsilon ; +\epsilon }$.
%
%Donc : pour tout $\epsilon > 0$, il existe un réel $A$ ($A=\frac1\epsilon$) tel que si $x>A$ alors $-\epsilon<\frac1x<\epsilon$. 
%\item $\lim\limits_{x\to +\infty}\frac 1x=0$ donc la droite d'équation $y=0$ est une asymptote horizontale à $\Crbf{f}$ en $+\infty$.
%\end{enumerate}
\end{mth}

\begin{mth}
Montrer que $\displaystyle \lim_{x\to +\infty} x^2=+\infty $.
%Soit $]A; +\infty [$, si $x>\sqrt A$ alors $x^2 >A$, \cad $x^2 \in ]A; +\infty [$.
\end{mth}

\begin{mth}
Déterminer $\displaystyle \lim_{1^-}\frac{x-3}{x^2-3x+2}$.
%$x^2-3x+2$ s'annule en 1, on factorise : $(x-1)(x-2)$.\\
%si $x<1$ alors $x-1<0$ et pour $x=1$, $x-2=-1$, ainsi $\displaystyle \lim_{x\to 1^-}(x-1)(x-2)=0^+$.\\
%Pour $x=1$, $x-3=-2$, (on a "$\frac{-2}{0^+}$") ainsi $\displaystyle \lim_{x\to 1^-}\frac{x-3}{x^2-3x+2}=-\infty$.
\end{mth}


\begin{mth}
Déterminer $\displaystyle \lim_{x \to +\infty}\sqrt{x^3-2x+1}$.
%On sait que $\displaystyle \lim_{x\to +\infty} x^3-2x+1=+\infty$, or $\displaystyle \lim_{X\to +\infty}\sqrt X=+\infty$, donc $\displaystyle \lim_{x \to +\infty}\sqrt{x^3-2x+1}=+\infty$.
\end{mth}